\chapter{Conclusion \& Future Work}
\label{ch:conclusion}
\section{Improvements Study Design}
\label{ImproveStudy}
The very promising findings from \fullref{ch:evaluation} are hampered by some of the limitations discussed in \fullref{sec:limitations}. It is likely that we had to few participants and not enough statistical power to detect a difference in trust and mistrust, however it is possible too that there is no influence on trust and mistrust between the conditions. To reduce the likelihood of a type II error more participants are necessary. As there was some indication that participants adjusted their responses between conditions a between subjects design might have yielded different results (single blind study). 
As larger sample sizes are more time consuming and costly it is important to assess whether a statistical significant change in mistrust or trust with a small effect size is necessary to detect at what costs:  Based on the effect sizes from the study (\fullref{sec:study}) the necessary sample sizes for significance with alpha error probability =.05 and power of =.80 would be about n=36 for Trust and n=93 for Mistrust (One tailed Wilcoxon signed-rank test with matched pairs). For a between subjects experiment (Wilcoxon-Mann-Whitney test) the a priori sample size is n=140 for Trust (70 per group) and n=368 for Mistrust (184 per group). The power analysis was performed with G*Power \cite{Faul2007GPower:Sciences.}. As discussed earlier (\fullref{ssec:questionaire},  Mistrust and Trust might not be the best indicators for an improvement of comfort as too many other factors influence it; recruiting 368 participants for 15 minute rides is thus absurd. 

Instead, for a small-scale study I recommend a within-subjects design with at least 18 participants who are properly sampled from the population and not acquainted with the researchers. This estimate is based upon the observed effect strength of the annoyance score; to be able to detect medium effect sizes (dz=0.5, \(\alpha\)=0.05, Power=0.80) 28 participants are necessary. Further, the measurements in \fullref{UXfactors} besides Trust are potentially more meaningful. Participants self-report measures can be flawed, thus more objective physiological measures should be taken into account.  Psychophysiological measures such as electroencephalography (EEG), cardiac changes (heart rate, blood pressure), ocular events (number and duration of eye blinks), changes in skin response, muscle activity (electromyography: EMG) and respiration \cite{Tichon2014PhysiologicalTraining} can be measured during the whole trip (and not just at one point of time) and then easier related to events. Nowadays not just low cost heart rate tracking wearables are accurate enough to measure stress through mean heart rate, low frequency and other simple time domain measures (pNN50, RMSSD, RR0) \cite{Salai2016}, but also relatively inexpensive wearables that can track galvanic skin response exist\footnote{\url{https://www.empatica.com/research/e4/}}\fnsep\footnote{\url{http://www.shimmersensing.com/products/shimmer3-wireless-gsr-sensor}}\fnsep\footnote{\url{https://www.movisens.com/en/products/eda-and-activity-sensor/}}. A good overview of techniques for stress estimation with heart rate and galvanic response wearables is provided in \cite{Ollander2015WearableEstimation}. Sometimes self-reported measures of affect (e.g. PANAS-X, MAACL-R, AD-ACL, UMACL) can be necessary for more subtle emotions that do not provoke a strong bodily response \cite{Boyle2015MeasuresDimensions}; MISC (MIsery SCore) is commonly used to measure car sickness (See \fullref{ssec:carsickness}.)

\section{Improvements Wizard Setup}
\label{ImproveWizard}
Driving Wizard and Interaction Wizard need to work in harmony. Some participants mentioned that some events activated too late for their liking (\fullref{ssec:interviews}). For some early prototyping this is sufficient, for a more sophisticated study the need for a interaction wizard should be eradicated. Surely the optimal study would take place in a fully autonomous vehicle without a safety driver on open roads. The open road is preferably over test tracks because of the amount of dynamic unforeseen events that can occur and the increased realism for external validity. As most companies and researchers still do not have access to these vehicles but also as most states do not issue licences for testing external subjects without a safety driver (\fullref{sec:study}) the effort for such user tests might be too high. Because the safety driver is required anyway there is no need for the vehicle to be autonomous. The scenario has to be believable to the test subjects and subjects should be prevented to see and interact with this safety driver as to not build up trust to him. The light message system should be automated as much as possible. With access to blinker settings, steering wheel angle, brake force and acceleration some of the messages can already be triggered by simple conditional statements if the driver drives foresighted (e.g. left/right, speed up). Other messages may require to set up GPS zones across the test track that trigger them (e.g. waiting for traffic light, arriving at destination). The few remaining light messages that cannot be pre-programmed could be triggered by buttons on the drivers steering wheel. Such a semi-automated system will deliver more consistent light messages which can help passengers to anticipate events more habitually and also increases internal study validity. If we eliminate the second wizard we can remove the curtain in front of the passenger and allow an increased field of view, which was a complaint from some study participants. The participant could then still be seated in the back and the driving wizard is comparted on his right like others did (\fullref{sec:wizard}) and his back how I did (\fullref{fig:testsetup}).  

\section{Improvements Ambient Display}
\label{ImrpoveDisplay}
The primal complaint from participants was that the ambient light was not bright enough. The \fullref{sec:videoAnalysis} showed that participants were alerted by it but the ambient display should work in even the brightness light settings. For the study luminance inside the cabin can be controlled with dimming film on the windows. The brightness of the production ready display should be dependent on the daylight: data can either be obtained from an internet database\footnote{\url{https://www.domoticz.com/wiki/Real-time_solar_data_without_any_hardware_sensor_:_azimuth,_Altitude,_Lux_sensor...}} or live from a LUX sensor attatched outside of the vehicle\footnote{\url{https://www.adafruit.com/product/439}}. To have live values would have the advantage that the light can be programmed to adjust automatically to dimmer light conditions in tunnels, parking garages etc. The maximum brightness of the ambient display can be increased if special light scattering lenses on top of the leds are used\footnote{\url{https://www.alibaba.com/product-detail/AT-diffus-pmma-led-lens-60_60452342195.html}}\fnsep\footnote{\url{https://www.ledsmagazine.com/articles/print/volume-14/issue-6/features/ssl-design/plastic-light-diffusion-systems-match-led-lighting-needs.html}} instead of other diffusing materials. The aesthetically preferable choice however is to use diffusing silicone rubber\footnote{\url{https://www.alibaba.com/product-detail/silicone-rubber-led-strip_60357176959.html}}\fnsep\footnote{\url{https://www.click-licht.de/198m-LED-Gabionen-Beleuchtung-Neon-Flex-Strip-warmweiss}} or an acrylic sheet that is placed with an offset to the LEDs (as planned in \fullref{fig:lasercut}) as in this case individual LEDs are not distinguishable. The LED strips used for the prototype had a lower density of 60LEDs per meter; LEDs with more than double the amount (144 pixels per meter\footnote{\url{https://www.aliexpress.com/item/1m-4m-5m-WS2812B-Smart-led-pixel-strip-Black-White-PCB-30-60-144-leds-m/2036819167.html}}) are common. If the framerate requirements are met (\fullref{LEDconsiderations}) more individually addressable led pixels would even increase the resolution. Hoewever, I think the resolution of the light display was sufficiently high enough and there is a theoretical limit on the framerate (\fullref{fig:FPS}). Therefore the easiest solution is to drive more strips in parallel that display equivalent light animations (connect the LED strips to the same data line). The logarithmic perception of lightness entails that 10 light strips would be required to give the impression that the light is twice as bright (\ref{sec:eye}). Cone cells, which are more densely distributed in the center of the eye, are more sensitive to colors, whereas rod cells, which are found concentrated at the outer edges of the retina, are more sensitive in general but not to color; this could explain why participants complained that light messages were too dim. possibly participants had issues to distinguish colors in peripheral vision but noticed the brightness changes and movement of the light. \say{Some of the colors were lost on me ... like red}
Another suggestment was to put the ambient display further down, into the pillars or below the windows. I would not recommend to do so as this ambient display is meant to turn the ceiling light into a lowcost multipurpose tool that provides supplementary information. Autonomous vehicles most likely have one or a few other displays to provide the navigation map anyway. 
The design of the light messages was well received. There were only a few suggestments: brake light should come from the back of the vehicle not from the front, colors for 'start' should be green and strong curves should be augmented as well. Finally the ambient display should provide an interface to the passengers to set the light options: color temperature and brightness. 
The light display was not meant to be the only communication channel, it was designed to keep the amount of necessary displays and sound feedback low, but be part of a multimodal feedforward information system. People will use their own devices in the autonomous car, in the case of a publicly shared vehicles they will order it via an app. This app can provide the route information and current location (as apps by Uber, MyTaxi and Lyft do now: \fullref{sec:interfaces}), thus the minimally configuration of an autonomous vehicle could have just the light display and speakers for more urgent feedback and feedback for the blind. A simple display with time to arrival and potentially a map can be helpful. I think the major advantage of a prominent display would be the possibility for technicians to dial in but also family members who have ordered the vehicle for family to dial in and have 'face to face contact'.

\section{Findings}
\label{sec:findings}
