\chapter{Introduction}
\label{ch:Introduction}

The word automobile originates from the French automobile, from ancient Greek autós (self) and French mobile (moving).  
Auto- + mobile, as an engine powers the vehicle rather than horses pull it\footnote{\url{https://en.wiktionary.org/wiki/automobile}\label{fn:wictionary}}. The concept of self-propelled vehicles is already described in the Iliad by 'Homer' (750 BC), by Leonardo da Vinci (15th century) and Swiss clergyman, J.H. Genevois (1760)\footnote{\url{https://www.britannica.com/technology/automobile/History-of-the-automobile}}. However, it was Karl Benz and Gottlieb Daimler who are attributed with the invention of the gasoline car and Henry Ford, who is attributed with mass-manufacturing vehicles through the invention of the assembly line, who kickstarted the worldwide success of the automobile\footnote{\url{https://en.wikipedia.org/wiki/Assembly_line}}. 

\emph{\say {L’automobile est un moyen de déplacement pratique à la campagne, mais cher et polluant.} - The automobile is a practical means of travel in the countryside, but it's expensive and polluting.\texorpdfstring{\textsuperscript{\ref{fn:wictionary}}}} 

One could argue that when the carriages lost their horse-drive, they became even less autonomous than before. Horses provided some intelligence that the motor-propelled vehicles do not have: today we would call them forward collision avoidance  (FCW) and lane keeping system (LKS). And indeed the danger of human drivers scared some\footnote{\url{https://timeline.com/forget-self-driving-car-anxiety-in-the-early-days-human-drivers-were-the-fear-55a770262c10}}, but ultimately due to the higher speed of automobiles carriages became obsolete as vehicles. Even today cars are expensive to maintain, polluting our environment with exhaust fumes and impractical in cities. The success of the car was so forceful that in fact, we turned new-built cities and towns into the countryside, the suburbs \citep{Forsyth2012}. This new car-focused lifestyle may be great for working mid-class families who want to raise their children in private safe spaces, but it can be disastrous for building communities, for singles and take independence from the elderly, the young, and the poor. 
Autonomous vehicles might help to free us from the car-obsessed world we created, but only if we design them right from the start to be inclusive, enjoyable and for the community. 